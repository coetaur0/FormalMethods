\documentclass[a4paper, titlepage]{article}
\usepackage[round, sort, numbers]{natbib}
\usepackage[utf8]{inputenc}
\usepackage{amsfonts, amsmath, amssymb, amsthm}
\usepackage{color}
\usepackage{listings}
\usepackage{marvosym}
\usepackage{mathtools}
\usepackage{paralist}
\usepackage{parskip}
\usepackage{subfig}
\usepackage{tikz}
\usepackage{titlesec}

\numberwithin{figure}{section}
\numberwithin{table}{section}

\usetikzlibrary{arrows, automata, backgrounds, petri, positioning}
\tikzstyle{place}=[circle, draw=blue!50, fill=blue!20, thick]
\tikzstyle{transition}=[rectangle, draw=black!50, fill=black!20, thick]

% define new commands for sets and tuple
\newcommand{\setof}[1]{\ensuremath{\left \{ #1 \right \}}}
\newcommand{\tuple}[1]{\ensuremath{\left \langle #1 \right \rangle }}
\newcommand{\card}[1]{\ensuremath{\left \vert #1 \right \vert }}

\makeatletter
\newcommand\objective[1]{\def\@objective{#1}}
\newcommand{\makecustomtitle}{%
	\begin{center}
		\huge\@title \\
		[1ex]\small Aurélien Coet, Dimitri Racordon
	\end{center}
	\@objective
}
\makeatother

\begin{document}

  \title{Outils formels de Modélisation \\ 4\textsuperscript{ème} séance d'exercices}
  \author{Aurélien Coet, Dimitri Racordon}
	\objective{
		Dans cette séance d'exercices, nous allons consolider notre étude des propriétés inhérentes aux réseaux de Petri.
    Notamment, nous étudierons les définitions formelles de ces propriétés.
	}

	\makecustomtitle

  \section{Vous avez dit formelles ... ($\bigstar\bigstar$)}
    Soit un réseau de Petri défini par $N=\tuple{P,T,^*\Delta,\Delta^*,M_0}$.
		Pour chacune des propriétés exprimées formellement ci-dessous, dessinez un exemple de réseau:
		\begin{enumerate}
			\item $|P|=3 \wedge \exists t \in T$ tq. $\forall p \in P: \Delta^*(t,p) = 1$
			\item $1 < \card{P} < \card{T} \wedge \forall p \in P: \forall t \in T: M \rightarrow^t M' \implies M'(p) > M(p)$
      \item $1 < \card{P} < \card{T} \wedge \exists s \in T^*$ tq. $M \rightarrow^s M' \implies M' \rightarrow^s M$
      \item $\forall s \in T^*: M \rightarrow^s M' \implies M \notin R(M') \wedge \nexists M' \in R(M_0)$ tq. $\forall p \in P: M'(p) = 0$
		\end{enumerate}

	\section{Electrocardiogramme [\Keyboard] ($\bigstar\bigstar\bigstar$)}
    Soit un réseau de Petri défini par $N=\tuple{P,T,^*\Delta,\Delta^*,M_0}$.
    On part de l'hypothèse que le réseau $N$ a nécessairement un nombre fini de marquages atteignables.
    
    \begin{enumerate}
        \item Écrivez un algorithme qui prend en entrée un réseau $N$ et une transition $t \in T$, et qui retourne vrai ou faux si $t$ est respectivement vivante ou non.
        \item Complétez la fonction \texttt{isAlive} dans le fichier \texttt{Analysis.fs} du projet \texttt{Exercise} avec une implémentation de votre algorithme.
        \item Le fichier \texttt{Program.fs} contient une implémentation du modèle vu en cours à la page 21 des slides \textit{Présentation des propriétés}. 
        Jouez avec le marquage initial de ce modèle et observez son influence sur la vivacité du réseau.
        Quel est la condition nécéssaire pour que le réseau soit vivant ? 
        \item Comment faudrait-il procéder pour que votre algorithme accepte en entrée un réseau avec un nombre de marquages potentiellement infini?
    \end{enumerate}

\end{document}
