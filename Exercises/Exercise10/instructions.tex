\documentclass[a4paper, titlepage]{article}
\usepackage[round, sort, numbers]{natbib}
\usepackage[utf8]{inputenc}
\usepackage{amsfonts, amsmath, amssymb, amsthm}
\usepackage{color}
\usepackage{listings}
\usepackage{mathtools}
\usepackage{multicol}
\usepackage{paralist}
\usepackage{parskip}
\usepackage{subfig}
\usepackage{tikz}
\usepackage{titlesec}

\numberwithin{figure}{section}
\numberwithin{table}{section}

\usetikzlibrary{arrows, automata, backgrounds, petri, positioning}
\tikzstyle{place}=[circle, draw=blue!50, fill=blue!20, thick]
\tikzstyle{marking}=[circle, draw=blue!50, thick, align=center]
\tikzstyle{transition}=[rectangle, draw=black!50, fill=black!20, thick]

% define new commands for sets and tuple
\newcommand{\setof}[1]{\ensuremath{\left \{ #1 \right \}}}
\newcommand{\tuple}[1]{\ensuremath{\left \langle #1 \right \rangle }}
\newcommand{\card}[1]{\ensuremath{\left \vert #1 \right \vert }}

\makeatletter
\newcommand\objective[1]{\def\@objective{#1}}
\newcommand{\makecustomtitle}{%
	\begin{center}
		\huge\@title \\
		[1ex]\small Aurélien Coet, Dimitri Racordon \\
	\end{center}
	\@objective
}
\makeatother

\begin{document}

  \title{Outils formels de Modélisation \\ 10\textsuperscript{ème} séance d'exercices}
  \author{Aurélien Coet, Dimitri Racordon}
	\objective{Dans cette séance d'exercices, nous allons étudier les bases de la logique propositionnelle, notamment en nous intéressant aux équivalences sémantiques.}

	\makecustomtitle

  \section{Tables de vérité ($\bigstar$)}
    Donnez les tables de vérité des formules logiques suivantes. Indiquez ensuite lesquelles de ces formules sont des tautologies.
    \begin{multicols}{2}
      \begin{enumerate}
        \item $\alpha \land \beta$
        \item $\alpha \implies \lnot\beta$
        \item $\alpha \land (\beta \lor \lnot\alpha)$
        \item $\alpha \land (\beta \lor \gamma)$
        \item $(\alpha \implies \beta) \lor (\alpha \lor \lnot\beta)$
        \item $(\alpha \lor \beta) \implies (\alpha \lor \gamma)$
      \end{enumerate}
    \end{multicols}

  \section{Transformations ($\bigstar\bigstar$)}
    Transformez les formules logiques suivantes en forme négative (NNF), en forme conjonctive (CNF) et en forme disjonctive (DNF).
    \begin{multicols}{2}
      \begin{enumerate}
        \item $\lnot(\alpha \land (\beta \lor \gamma))$
        \item $(\alpha \implies \beta) \lor \lnot(\alpha \land \gamma)$
        \item $(\lnot\alpha \lor \beta \land \gamma) \land \alpha$
      \end{enumerate}
    \end{multicols}

  \section{Equivalences logiques ($\bigstar\bigstar\bigstar\bigstar$)}
    Démontrez les équivalences logiques suivantes.
    \begin{enumerate}
      \item $(\alpha \lor \beta) \land (\lnot\alpha \lor \gamma)
              \equiv (\lnot\alpha \land \beta) \lor (\gamma \land (\alpha \lor \beta))$
      \item $(\alpha \lor \beta) \implies \lnot\gamma
              \equiv \lnot((\alpha \land \gamma) \lor (\beta \land \gamma))$
    \end{enumerate}

  \section{Exclusivité ($\bigstar\bigstar\bigstar$)}
    Donnez une formule de logique propositionnelle équivalente à la ligne de code suivante,
    écrite en F\#:
    \begin{verbatim}
      let c = if a <> b then true else false
    \end{verbatim}

\end{document}
